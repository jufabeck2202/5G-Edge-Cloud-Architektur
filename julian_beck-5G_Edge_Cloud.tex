\documentclass[runningheads]{llncs}
%---- Sonderzeichen-------%
\usepackage[ngerman]{babel}
%---- Codierung----%
\usepackage[utf8]{inputenc}
%\usepackage[latin1]{inputenc}
\usepackage[T1]{fontenc}
\usepackage{graphicx}
\usepackage{url}
\usepackage{llncsdoc}
%----- Mathematischer Zeichenvorrat---%
\usepackage{amsmath}
\usepackage{amssymb}
\usepackage{enumerate}
% fuer die aktuelle Zeit
\usepackage{listings}
\usepackage{textcomp}
\usepackage{color}
\usepackage{subfigure}
\usepackage{hyperref}
\numberwithin{figure}{section}

\usepackage[citestyle=numeric,style=numeric,backend=biber]{biblatex}
\addbibresource{literature.bib}

%
% ECMAScript 2015 (ES6) definition by Gary Hammock
%

\lstdefinelanguage[ECMAScript2015]{JavaScript}[]{JavaScript}{
  morekeywords=[1]{await, async, case, catch, class, const, default, do,
    enum, export, extends, finally, from, implements, import, instanceof,
    let, static, super, switch, throw, try},
  morestring=[b]` % Interpolation strings.
}


%
% JavaScript version 1.1 by Gary Hammock
%
% Reference:
%   B. Eich and C. Rand Mckinney, "JavaScript Language Specification
%     (Preliminary Draft)", JavaScript 1.1.  1996-11-18.  [Online]
%     http://hepunx.rl.ac.uk/~adye/jsspec11/titlepg2.htm
%

\lstdefinelanguage{JavaScript}{
  morekeywords=[1]{break, continue, delete, else, for, function, if, in,
    new, return, this, typeof, var, void, while, with},
  % Literals, primitive types, and reference types.
  morekeywords=[2]{false, null, true, boolean, number, undefined,
    Array, Boolean, Date, Math, Number, String, Object},
  % Built-ins.
  morekeywords=[3]{eval, parseInt, parseFloat, escape, unescape},
  sensitive,
  morecomment=[s]{/*}{*/},
  morecomment=[l]//,
  morecomment=[s]{/**}{*/}, % JavaDoc style comments
  morestring=[b]',
  morestring=[b]"
}[keywords, comments, strings]


\lstalias[]{ES6}[ECMAScript2015]{JavaScript}

% Requires package: color.
\definecolor{mediumgray}{rgb}{0.3, 0.4, 0.4}
\definecolor{mediumblue}{rgb}{0.0, 0.0, 0.8}
\definecolor{forestgreen}{rgb}{0.13, 0.55, 0.13}
\definecolor{darkviolet}{rgb}{0.58, 0.0, 0.83}
\definecolor{royalblue}{rgb}{0.25, 0.41, 0.88}
\definecolor{crimson}{rgb}{0.86, 0.8, 0.24}

\lstdefinestyle{JSES6Base}{
  backgroundcolor=\color{white},
  basicstyle=\ttfamily,
  breakatwhitespace=false,
  breaklines=false,
  captionpos=b,
  columns=fullflexible,
  commentstyle=\color{mediumgray}\upshape,
  emph={},
  emphstyle=\color{crimson},
  extendedchars=true,  % requires inputenc
  fontadjust=true,
  frame=single,
  identifierstyle=\color{black},
  keepspaces=true,
  keywordstyle=\color{mediumblue},
  keywordstyle={[2]\color{darkviolet}},
  keywordstyle={[3]\color{royalblue}},
  numbers=left,
  numbersep=5pt,
  numberstyle=\tiny\color{black},
  rulecolor=\color{black},
  showlines=true,
  showspaces=false,
  showstringspaces=false,
  showtabs=false,
  stringstyle=\color{forestgreen},
  tabsize=2,
  title=\lstname,
  upquote=true  % requires textcomp
}

\lstdefinestyle{JavaScript}{
  language=JavaScript,
  style=JSES6Base
}
\lstdefinestyle{ES6}{
  language=ES6,
  style=JSES6Base
}
\renewcommand{\labelitemi}{$\bullet$}
%\renewcommand{\thefigure}{\thesection-\arabic{figure}}

\setcounter{tocdepth}{3}
\setcounter{secnumdepth}{3}

% -------------------------------------------------------------------------------------------------
% -------------------------------------------------------------------------------------------------
\mainmatter
\title{5G Edge Cloud Architektur}
\titlerunning{5G Edge Cloud Architektur}
\author{Julian Beck}
\authorrunning{Julian Beck}
\institute{Betreuer: Prof. Dr. rer. nat. Oliver Waldhorst}
\date{01.05.2019}
\begin{document}
\let\oldaddcontentsline\addcontentsline
\def\addcontentsline#1#2#3{}
\maketitle
\def\addcontentsline#1#2#3{\oldaddcontentsline{#1}{#2}{#3}}


% -------------------------------------------------------------------------------------------------

\begin{abstract}
  Die Edge Cloud wird in dem Zeitalter von 5G eine wichtige Rolle spielen. 
  Als ein Bestandteil der 5G-Netzwerkarchitektur bietet es nicht nur eine Vielzahl von Cloud-Ressourcen, sondern
  ermöglicht neue Platformen für Drittanbieter und das Entwickeln von neuen Erfahrungen für den Nutzer.
  Multi-Access Edge Computing (MEC) bietet Speicher- und Rechenressourcen in der Nähe des Endgerätes, 
  eine besser Latenzzeit für mobile Endbenutzer und effizientere Nutzung des Mobile Backhaul
  und Core Netzwerkes. Diese Seminararbeit erläutert, welche Technologien MEC ermöglicht und geht auf die Architekturen 
  hinter Multi-Access Edge Computing ein.
\end{abstract}

% -------------------------------------------------------------------------------------------------
\tableofcontents 
\newpage
% -------------------------------------------------------------------------------------------------

\section{Einleitung}
\label{sec:Einleitung}
\subsection{Problem}
\label{subsec:Problem}
\section{Edge Computing}
\label{sec:Edge Computing}
Beim Edge Computing wird die Berechnung und Speicherung von Daten in die Nähe der Quelle gebracht,
an den sogenannten Rand oder \textit{Edge} des Netzwerks. Im Gegensatz zum Cloud Computing werden die Daten
nicht an zentralen Rechenzentren verarbeitet, sonden an dezentralen Cloud Systemen am Rand des Neztwerks. 
Folgende Vorteile bringt Edge Computing: \cite{labrieTopBenefitsEdge}
\begin{itemize}
  \item \textbf{Geschwindigkeit und Latenz:} Abhängig von der Anwendungen spielt die Länge der Datenverarbeitung 
  Entscheidende Rolle. Beispielsweise bei Autonomen Fahrzeugen ist es wichtig, das innerhalb von Millisekunden die Daten
  verarbeitet werden. Auch bei Digitalen Fabriken ist es meist zu langsam die Daten zu einer zentralen Cloud und zurück
  zu senden. 
  Wenn die Datenverarbeitung auf den Rand des Netzwerks verlegt wird, wird die Latenz des Netzwerks verhindert und schneller
  auf Anfragen geantwortet.
  \item \textbf{Netzlast:} Dadurch das die Daten nicht zu einer zentralen Cloud gesendet werden, sonder am Rand des Netzwerks 
  verarbeitet werden, verringert sich nicht nur die Latenz sonder auch die Netzlast des gesamten Netzwerks. 
  Die Daten müssen nicht weitreichend weiter gesendet werden, sondern werden dezentral in der Nähe der Anwendungen verarbeitet.
  \item \textbf{Security:} Wenn Daten an einem zentralen Cloud verarbietet werden ist diese sehr Anfällig. 
  So kann Beispielsweise ein DDoS-Angriff den gesammten Betrieb eines Unternehmens stören wenn alle Systeme mit einer zentralen
  Cloud Arbeiten. Da bei Edge Computing kein einziges Zentrales Systeme existiert, verringert sich die Auswirkung eines solchen
  Angriffes auf das gesamte unternehmen.  
  Edge Computing hilft Unternehmen auch dabei, die Probleme der lokalen Compliance- und Datenschutzbestimmungen zu überwinden,
  da die Daten auf lokalen Systemen verarbeitet werden.
  \item \textbf{Kosteneinsparungen:} Durch Internet Off Things Geräte oder auch Beispielsweise durch eine Smart Factory wird
  eine vielzahl an Daten generiert. Nicht alle Daten sind dabei kritisch für die Operation der Systeme. Edge Computing erlaubt
  das Kategorisieren der Daten. In dem ein großteil der Verarbeitung am Rand des Netzwerks stattfindet wird Bandbreite gespart.
  eim Edge-Computing geht es nicht darum, die Notwendigkeit der Cloud zu beseitigen, sondern darum, den Datenfluss zu optimieren, 
  um Ihre Betriebskosten zu maximieren.
  \item \textbf{Zuverlässigkeit:} Wenn Edge-Geräte Daten lokal speichern und verarbeiten können, verbessert dies die Zuverlässigkeit.
  Ein Unternehmen ist nicht auf die Verbindung auf zur zentralen Cloud angewiesen. 
  Eine vorübergehende Unterbrechungen der Konnektivität hat keine Auswirkungen auf den Bettrieb von Geräte,
  nur weil sie die Verbindung zur Cloud verloren haben.
  \item \textbf{Skalierbarkeit:}
  Bei Cloud-Computing-Architekturen müssen Daten in den meisten Fällen zunächst an ein zentral gelegenes Rechenzentrum
  weitergeleitet werden. Das Erweitern oder sogar nur das Ändern dedizierter Rechenzentren ist eine teure Angelegenheit. 
  Darüber hinaus können IoT-Geräte zusammen mit ihren Verarbeitungs- und Datenverwaltungstools am Rande einer einzelnen 
  Implantation bereitgestellt werden, 
  anstatt auf die Koordination der Bemühungen von Mitarbeitern an mehreren Standorten zu warten.
\end{itemize}
\newpage

\section{5G Edge Computing}
\label{sec:5G Edge Computing}

\newpage
\section{Architekturen}
\label{sec:Architekturen}
\subsection{Standards}
\label{subsec:Architekturen}

\section{Anwendungen}
\label{sec:Anwendungen}
\subsection{Internet of Things}
\label{subsec:Internet of Things}
\subsection{Smart Factory}
\label{subsec:Smart Factory}
\subsection{Autonomes Fahren}
\label{subsec:Autonomes Fahren}
\section{Ausblick}
\label{sec:Ausblick}

% -------------------------------------------------------------------------------------------------

% -------------------------------------------------------------------------------------------------
\newpage
% Normaler LNCS Zitierstil
%\bibliographystyle{splncs}
\printbibliography[heading=bibintoc]


\end{document}
