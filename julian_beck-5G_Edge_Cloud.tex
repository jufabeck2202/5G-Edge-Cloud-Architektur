\documentclass[runningheads]{llncs}
%---- Sonderzeichen-------%
\usepackage[ngerman]{babel}
%---- Codierung----%
\usepackage[utf8]{inputenc}
%\usepackage[latin1]{inputenc}
\usepackage[T1]{fontenc}
\usepackage{graphicx}
\usepackage{url}
\usepackage{llncsdoc}
%----- Mathematischer Zeichenvorrat---%
\usepackage{amsmath}
\usepackage{amssymb}
\usepackage{enumerate}
% fuer die aktuelle Zeit
\usepackage{listings}
\usepackage{textcomp}
\usepackage{color}
\usepackage{subfigure}
\usepackage{hyperref}
\numberwithin{figure}{section}

\usepackage[citestyle=numeric,style=numeric,backend=biber]{biblatex}
\addbibresource{literature.bib}

%
% ECMAScript 2015 (ES6) definition by Gary Hammock
%

\lstdefinelanguage[ECMAScript2015]{JavaScript}[]{JavaScript}{
  morekeywords=[1]{await, async, case, catch, class, const, default, do,
    enum, export, extends, finally, from, implements, import, instanceof,
    let, static, super, switch, throw, try},
  morestring=[b]` % Interpolation strings.
}


%
% JavaScript version 1.1 by Gary Hammock
%
% Reference:
%   B. Eich and C. Rand Mckinney, "JavaScript Language Specification
%     (Preliminary Draft)", JavaScript 1.1.  1996-11-18.  [Online]
%     http://hepunx.rl.ac.uk/~adye/jsspec11/titlepg2.htm
%

\lstdefinelanguage{JavaScript}{
  morekeywords=[1]{break, continue, delete, else, for, function, if, in,
    new, return, this, typeof, var, void, while, with},
  % Literals, primitive types, and reference types.
  morekeywords=[2]{false, null, true, boolean, number, undefined,
    Array, Boolean, Date, Math, Number, String, Object},
  % Built-ins.
  morekeywords=[3]{eval, parseInt, parseFloat, escape, unescape},
  sensitive,
  morecomment=[s]{/*}{*/},
  morecomment=[l]//,
  morecomment=[s]{/**}{*/}, % JavaDoc style comments
  morestring=[b]',
  morestring=[b]"
}[keywords, comments, strings]


\lstalias[]{ES6}[ECMAScript2015]{JavaScript}

% Requires package: color.
\definecolor{mediumgray}{rgb}{0.3, 0.4, 0.4}
\definecolor{mediumblue}{rgb}{0.0, 0.0, 0.8}
\definecolor{forestgreen}{rgb}{0.13, 0.55, 0.13}
\definecolor{darkviolet}{rgb}{0.58, 0.0, 0.83}
\definecolor{royalblue}{rgb}{0.25, 0.41, 0.88}
\definecolor{crimson}{rgb}{0.86, 0.8, 0.24}

\lstdefinestyle{JSES6Base}{
  backgroundcolor=\color{white},
  basicstyle=\ttfamily,
  breakatwhitespace=false,
  breaklines=false,
  captionpos=b,
  columns=fullflexible,
  commentstyle=\color{mediumgray}\upshape,
  emph={},
  emphstyle=\color{crimson},
  extendedchars=true,  % requires inputenc
  fontadjust=true,
  frame=single,
  identifierstyle=\color{black},
  keepspaces=true,
  keywordstyle=\color{mediumblue},
  keywordstyle={[2]\color{darkviolet}},
  keywordstyle={[3]\color{royalblue}},
  numbers=left,
  numbersep=5pt,
  numberstyle=\tiny\color{black},
  rulecolor=\color{black},
  showlines=true,
  showspaces=false,
  showstringspaces=false,
  showtabs=false,
  stringstyle=\color{forestgreen},
  tabsize=2,
  title=\lstname,
  upquote=true  % requires textcomp
}

\lstdefinestyle{JavaScript}{
  language=JavaScript,
  style=JSES6Base
}
\lstdefinestyle{ES6}{
  language=ES6,
  style=JSES6Base
}
\renewcommand{\labelitemi}{$\bullet$}
%\renewcommand{\thefigure}{\thesection-\arabic{figure}}

\setcounter{tocdepth}{3}
\setcounter{secnumdepth}{3}

% -------------------------------------------------------------------------------------------------
% -------------------------------------------------------------------------------------------------
\mainmatter
\title{5G Edge Cloud Architektur}
\titlerunning{5G Edge Cloud Architektur}
\author{Julian Beck}
\authorrunning{Julian Beck}
\institute{Betreuer: Prof. Dr. rer. nat. Oliver Waldhorst}
\date{01.05.2019}
\begin{document}
\let\oldaddcontentsline\addcontentsline
\def\addcontentsline#1#2#3{}
\maketitle
\def\addcontentsline#1#2#3{\oldaddcontentsline{#1}{#2}{#3}}


% -------------------------------------------------------------------------------------------------

\begin{abstract}
  Die Edge Cloud wird in dem Zeitalter von 5G eine wichtige Rolle spielen. 
  Als ein Bestandteil der 5G-Netzwerkarchitektur bietet es nicht nur eine Vielzahl von Cloud-Ressourcen, sondern
  ermöglicht neue Platformen für Drittanbieter und das Entwickeln von neuen Erfahrungen für den Nutzer.
  Multi-Access Edge Computing (MEC) bietet Speicher- und Rechenressourcen in der Nähe des Endgerätes, 
  eine besser Latenzzeit für mobile Endbenutzer und effizientere Nutzung des Mobile Backhaul
  und Core Netzwerkes. Diese Seminararbeit erläutert, welche Technologien MEC ermöglicht und geht auf die Architekturen 
  hinter Multi-Access Edge Computing ein.
\end{abstract}

% -------------------------------------------------------------------------------------------------
\tableofcontents 
\newpage
% -------------------------------------------------------------------------------------------------

\section{Einleitung}
\label{sec:Einleitung}
In den letzten zehn Jahren haben Fortschritte im Cloud-Computing einen zentralisierten Ansatz für die Systemadministration und den 
Systembetrieb verfolgt, während das Wachstum von Mobile Computing, 
SaaS und dem Internet der Dinge (IoT) das Computing in Richtung einer verteilten Architektur in der Nähe des Anwenders getrieben hat. 
Mit der Einführung von 5G-und Edge-Computing-Technologien möchten Unternehmen nun beide Ansätze kombinieren.
\\
\\
5G und Edge Computing sind zwei relativ neue Technologien, die aber ähnliche Ziele verfolgen. 
Beide sind darauf ausgerichtet, die Leistung von Anwendungen zu verbessern 
und die Verarbeitung von großer Datenmengen in Echtzeit zu ermöglichen. 
5G erhöht die Geschwindigkeit um das Zehnfache gegenüber 4G, 
während  Edge Computing die Latenz verringert, indem Rechenfunktionen näher am Endbenutzer in das Netzwerk integriert werden.
\\
\\
Während Telekommunikationsbetreiber berichten, 
dass 5G im Netzwerkgeschwindigkeiten liefern kann, die mehr als zwanzigmal schneller sind als LTE1, 
spiegelt dies nicht die Erfahrung eines durchschnittlichen Benutzers wieder. Die Edge Cloud kommt an dieser stelle ins Spiel und verbessert die Latanz und 
damit auch die Anwendungsmöglichkeiten von 5G.
Gleichzeitig benötigt 5G das Edge Computing, um die Nachfrage zu steigern. 
\subsection{Problem}
\label{subsec:Problem}
5G Edge Computing bietet sich für folgende Anwendungen an
\begin{itemize}
  \item Caching von Anwendungen und Videos
  \item Rechenressourcen an die Aufgaben übergeben werden können um möbile Endgeräte zu entlasten
  \item Bearbeitung und Aggregierung von IoT Daten.
  \item Rendern von Videospielen auf der EDGE Cloud.
\end{itemize}
Die Anwendungsbeispiele werden in Kaptiel \ref{sec:Anwendungen} genauer erläutert.
\section{Benötigte Technologien}
\label{sec:Benötigte Technologien}
\subsection{Edge Computing}
\label{sub:Edge Computing}
Bei Cloud-Computing werden Rechenressourcen über ein Netzwerk zu Verfügung gestellt.
Beim Edge Computing wird die Berechnung und Speicherung von Daten in die Nähe der Quelle gebracht,
an den sogenannten Rand oder \textit{Edge} des Netzwerks. Im Gegensatz zum Cloud Computing werden die Daten
nicht an zentralen Rechenzentren verarbeitet, sonden an dezentralen Cloud Systemen am Rand des Neztwerks. 
Folgende Vorteile bringt Edge Computing: \cite{labrieTopBenefitsEdge}
\begin{itemize}
  \item \textbf{Geschwindigkeit und Latenz:} Abhängig von der Anwendungen spielt die Zeit der Datenverarbeitung eine
  entscheidende Rolle. Beispielsweise bei Autonomen Fahrzeugen ist es wichtig, dass innerhalb von Millisekunden die Daten
  verarbeitet werden. Auch bei digitalen Fabriken ist es meist zu langsam die Daten zu einer zentralen Cloud und zurück
  zu senden. 
  Wenn die Datenverarbeitung auf den Rand des Netzwerks verlegt wird, wird die Latenz des Netzwerks verringert und schneller
  auf Anfragen geantwortet.
  \item \textbf{Netzlast:} Da das die Daten nicht zu einer zentralen Cloud gesendet werden, sonder am Rand des Netzwerks 
  verarbeitet werden, verringert sich nicht nur die Latenz, sondern auch die Netzlast des gesamten Netzwerks. 
  Die Daten müssen nicht weitreichend weiter gesendet werden, stattdessen werden sie 
  dezentral in der Nähe der Anwendungen verarbeitet.
  \item \textbf{Security:} Wenn Daten an einem zentralen Cloud verarbietet werden ist dies unter bestimmten Umständen anfälliger
  für ein Ausfall.
  So kann beispielsweise ein DDoS-Angriff den gesammten Betrieb eines Unternehmens stören, wenn alle Systeme mit einer zentralen
  Cloud arbeiten. Da bei Edge Computing kein einziges Zentrales Systeme existiert, verringert sich die Auswirkung eines solchen
  Angriffes für das ganze Unternehmen.  
  Edge Computing hilft Unternehmen auch dabei, die Probleme der lokalen Compliance- und Datenschutzbestimmungen zu überwinden,
  da die Daten auf lokalen Systemen verarbeitet werden.
  \item \textbf{Kosteneinsparungen:} Durch Internet Off Things Geräte oder durch eine Smart Factories werden
  eine Vielzahl an Daten generiert. Nicht alle Daten sind dabei kritisch für die Operation der Systeme. Edge Computing erlaubt
  das Kategorisieren der Daten. In dem ein Großteil der Verarbeitung am Rand des Netzwerks stattfindet wird Bandbreite gespart.
  Dies optimiert den Datenfluss von lokalen Anwendungen und minimiert somit die Betriebskosten einer zentralen Cloud.
  \item \textbf{Zuverlässigkeit:} Wenn Edge-Geräte Daten lokal speichern und verarbeiten können, verbessert dies die Zuverlässigkeit.
  Ein Unternehmen ist nicht auf die Verbindung auf zur zentralen Cloud angewiesen und eine eine vorübergehende Unterbrechungen der 
  Verbingung hat keine Auswirkungen auf den Betrieb von Geräte, nur weil sie die Verbindung zur Cloud verloren haben.
  \item \textbf{Skalierbarkeit:}
  Bei Cloud-Computing-Architekturen müssen Daten in den meisten Fällen zunächst an ein zentral gelegenes Rechenzentrum
  weitergeleitet werden. Das Erweitern oder sogar nur das Ändern dedizierter Rechenzentren ist eine teure Angelegenheit. 
  Darüber hinaus können IoT-Geräte zusammen mit ihren Verarbeitungs- und Datenverwaltungstools am Rande einer einzelnen 
  Implantation bereitgestellt werden, 
  anstatt auf die Koordination der Bemühungen von Mitarbeitern an mehreren Standorten zu warten.

\end{itemize}
\newpage

\subsection{5G Edge Computing}
\label{subsec:5G Edge Computing}
\subsection{Network-Function Virtualisierung}
\label{subsec:Network-Function Virtualisierung}
Network-Function Virtualisierung (NFV) erlaubt es Netzwerk Funktionen von der Hardware zu entkoppeln.
Dies erlaubt das Verwenden von Gateways, Firewalls, DNS Services und Caching ohne proprietär Hardware.
\\
\\
Um ein neues Netwzerk zu erstellen, wird eine Vielzahl von verschieden Hardware Komponenten benötigt. 
Diese benötigen Platz, Energy und müssen von qualifizierten Personal überwacht und gewartet werden. 
Network-Function Virtualisierung will diese Probleme lösen, in dem Virtualisierungstechniken auf standard
Server Hardware Komponenten verwendet werden. 
Foldgende Vorteile werden durch NFV erziehlt und sind Relevant 5G Edge Clouds: \cite{nfv_wp}
\begin{itemize}
  \item \textbf{Skalierbarkeit und Flexibilität:} Die Virtualisierung erlaubt eine einfache skallierung der Ressourcen.
  So kann bei einer großen Nachfrage die Services skalliert werden. Es können auch schnell mehrere Instanzen einer Komponente auf 
  der einer VM gestartet werden.
  \item \textbf{Kosteneinsparungen:} Durch die Verwendung von Standard Komponenten werden die Kosten und der Energieverbrauch minimiert.
  \item \textbf{Anpassungsfähigkeit:} Die Virtualisierung erlaubt eine schnelle Anpassung an Anforderungen eines Kunden. 
  Die Server können durch der Virtualiserung von mehreren Nuzern gleichzeitig verwendet werden und gleichzeitig die Anforderungen des jeweiligen
  Kunden erfüllen.
  \item \textbf{Verbesserte Effizienz durch homogene Systeme:}
\end{itemize}
\subsubsection{NFV Architektur und Orchestration Framework:}
Das Europäische Institut für Telekommunikationsnormen \textit{ETSI} hat ein Standard für ein NFV Framework veröffentlicht.
Dieser Standard definiert sogenannte Virtualized Network Functions \textit{VNF}, welche Neztwek Funktionen in Software abbilden.
Die \textit{VNFs} werden in der \textit{NFV} Infrastruktur \textit{NFVI} deployed. Die \textit{NFV} Infrastruktur besteht aus den
enthält die Hardware Komponenten wie CPU und Speicher, aber auch die Virtualisierungslayer. \\
Der \textit{NFV MANO} (NFV Managment und Orchestrierung) Layer verwaltet die Infrastruktur und passt diese an die Anforderungen an.
Der VM lifevyvle wird auch von \textit{NFV MANO} verwaltet, wenn eine VM abstürtr wird sie von diesem neugestartet.
\subsubsection{5G Edge Cloud und NFV:}
Die Network-Function Virtualisierung spielt eine Schlüsselrolle für die Umsetzung einer 5G Edge Cloud.
\textit{NFV} erlaubt \textit{Network Slicing}, ein Aspekt der virtuellen Netzwerkarchitektur, 
mit dem mehrere virtuelle Netzwerke auf einer gemeinsam genutzten Infrastruktur bereitgestellt werden können.
\textit{NFV} ermöglicht die 5G-Virtualisierung, sodass das physische Netzwerk in mehrere virtuelle Netzwerke unterteilt werden können. 
Dies erlaubt es unterschiedliche Radio Access Networks ()\textit{RAN}) 
oder verschiedene Arten von Diensten gleichzeitig anzubieten. Der Anwender merkt dabei kein Unterschied, da die Network Slices
voneinander isoliert sind. 
\textit{NFV} ist für die Skalierbarkeit, Flexibilität und Migration in einer 5G Edge Cloud wichtig. So kann wenn die Anfragen an eine
Anwendung steigt, nicht nur die Ressourcen für die Anwendung an sich einfach skaliert werden, sondern durch das hinzufügen einer neuen
Software Instant in der \textit{NFVI}, kann auch die Netzwerkinfrastruktur mit skaliert werden. \cite{How5GNFV}
\subsection{Software-defined Networking}
\label{subsec:Software-defined Networking}
Neben \textit{NFV}, ist auch Software Defined Networking \textit{(SDN)} ein Schlüssel für Virtualisierung von Netzwerken.
Bei einem \textit{SDN} ist die Steuerung des Netzwerks von der Hardware getrennt. Es wird zwischen einem Controller, einer Southbound und einer
Northbound API unterschieden. Die Southbound APIs führt die Anweisungen nimmt und gibt Informationen des Controllers weiter an Netzwerkgeräte wie Switches,
Access Points und Router weiter. Der Controller ist das zentrale Element eines \textit{SDN} Netzwerkes, er ermöglicht ein zentrales managen und steuern des 
Netzwerkes. Die Northbound API gibt Informationen an den Controller weiter. Es ist die schnittstelle zwiscen Anwendungen und dem \textit{SDN} Controller.\cite{SoftwareDefinedNetworkingSDN}
\\
\\
Software Defined Networking kann so \textit{MEC} unterstützen, indem es automatisch und flexibel service management durchführt.
Da bei einem SDN die Daten und Controll Ebene durch die Southbound und Norhtbound API getrennt ist, 
führt SDN eine zentrale Steuerung ein, 
mit der virtuelle Netzwerkinstanzen einfach instanziiert und angeboten werden können, 
indem die zugrunde liegende Netzwerkinfrastruktur abstrahiert wird. \\
Im Kontext von MEC kann der SDN-Controller MEC-bezogene VNFs, 
VMs und Container als eine weitere Netzwerkkomponente behandeln, der dynamisch zugewiesen und neu lokalisiert werden kann.
So kann der SDN flexibel Service anpassen und dynamsich Dienste bereitstellen, indem er VNFs und MEC Dienste verbindet.
Gleichzeitig kann er die mobilität der Dienste ermöglichen.
\\
\\
Die Kombination von SDNs und NFV erlaubt das erstellen von Network Slices. 

\subsection{Virtuelle Maschinen und Container}
\label{subsec:Virtuelle Maschinen und Container}
Eine Cloud Platform besteht typischerweise aus einer Anzahl an Maschinen die durch ein Hypervisor zu einer zentralen Maschine zusammen gefasst werden.
Dieser kann isolierte Virtuelle Maschinen erstellen und ausführen und dient als Abstraktionsebene, unabhängig von der Hardware auf denen die VMs laufen.
\\
\\
Eine leichgewichtige Alternative zur Hypervisor-basierten Virtualisierung ist die containergestützte Virtualisierung. 
Diese im Vergleich zu anderen Virtualisierungslösungen eine andere  Abstraktionsebene in Bezug auf Virtualisierung und Isolation. 
Container implementieren die Isolierung auf OS Ebene und vermeiden so die Virtualisierung von Hardware und Treibern. I
nsbesondere teilen sich Container denselben Kernel mit dem zugrunde liegenden Hostcomputer. Dies macht Container sehr leichtgewichtigt und flexibel im 
Gegensatz zu VM. Typischerweise führt ein Container genau ein Service aus, was eine schnelle Migration ermöglicht.
\\
\\
Im Blick auf \textit{MEC} ermöglichen Container eine leichtgewichtige Virtualisierungslösung. So eigenen sich die Container
als eine portable Laufzeit Umgebung für MEC-Dienste. Einzelne Dienste können in Containern ausgeführt werden und sind so isoliert und können einfach 
verwaltet und gesteuert werden. Für die Umsetzung bietet sich die mitlerweile weitverbreitete Containerlösung Docker an. Hier stehen mit 
Kubernetes auch Orchestrierungs und Clustering Tools zur Verfügung  \cite{morabitoConsolidateIoTEdge2018}
\newpage
\section{MEC Framework - Referenz Architektur}
\label{subsec:MEC Framework - Referenz Architektur}
Folgendes Kaptiel zeigt die dem European Telecommunications Standards Institute (ETSI) beschriebene
Referenz Architektur zur Implementation eines \textit{MEC} Systems. 
\\ 
\\
\textit{ETSI} beschreibt dabei in ihrer Spezifikation \textit{Multi-access Edge Computing (MEC); 
Framework and Reference Architecture} \cite{etsiETSIGSMEC} ein Framework und eine 
Referenz Architektur.
\subsection{Multi-access Edge Computing Framework}
Das Framework unterscheidet die einzelnen Komponenten die für die Edge Cloud benötigt werden in drei Ebenen. 
Die Komponenten werden dabei auf Infrastruktur virtualisiert am Rande des Neztwerks ausgeführt. 
Dies ist möglich durch die in Kapitel \ref{sec:Benötigte Technologien} vorgestellten Technologien.
Die Grafik X zeigt die komponenten des Frameworks

\subsubsection{Netzwerk Ebene:}
Die unterste Ebene des Frameworks ist die Netzwerk Ebene. Sie ermöglicht die verbindung zu dem lokalen und 
externen Netzwerk.  Die 3GPP Komponente steht für 3rd Generation Partnership Project, was ein überbegriff für die mobieln
Telecommunikationsstandrads sind wie LTE und 5G. 
\subsubsection{MEC Host}
Der MEC Host enthält die MEC Platform und die Infrastruktur auf den die Anwendungen in der Edge Cloud betreiben werden.
Die Inrastruktur stellt Rechen-, Speicher- und Netzwerkressourcen  virtualisert zu verfügung. Als Virtualisierungslösung kann
die in Kapitel \ref{subsec:Network-Function Virtualisierung} eingeführte Network-Function Virtualisierung Infrastruktur
eingesetzt werden. der MEC Host besteht aus zwei weiteren Komponenten:
\\
\\
\textbf{MEC Platform:} Die MEC Platform dient als eine art Registry 
Anwendungen die auf der MEC Infrastruktur laufen. Die Platform bietet eine Umgebung, in der MEC Anwendungen sich anmelden können,
andere Anwendungen Anfragen können. Die Platform ist auch für DNS Reccords zuständig. Die MEC Platform nimmt Befehle
des MEC Platform Managers entgegen und passt die DNS Records, Proxies an. Desweiteren verwältet er auch den Persistent Storage.
\\
\\
\textbf{MEC Anwendungen:} Multi-Access Edge Computing Anwendungen werden in einer Virtuellen Maschine oder Als
Container auf der Infrastruktur des MEC hosts ausgeführt. Die Anwendungen registireren sich bei der MEC Platform
Komponente um darüber andere Anwendungen anzufragen und ihren Service zur verfügung zu stellen.

\subsubsection{MEC host level management:}
Auf der Host Ebene ist auch die MEC Host Level Management Komponente. Ähnlich wie der MEC Host, ist diese in weitere 
Komponenten aufgeteilt:
\\
\\
\textbf{MEC Platform Manager:} Der Platform Manager ist verwaltet den Lifecycle der Anwendungen. Er startet, stoppt und neustartet diese.
Gleichzeitig informiert er den Multi-Access edge Orchestrator über wichtige Events der Anwendungen. 
Eine weitere Aufgabe ist das verwalten und managen von autorisierung und DNS Configurationen und Problemen.
Desweiteren erhölt der Platform Manager Fehlerberichte und Informationen über die Infrastruktur von dem Virtualiserungs Infrastruktur Manager.
\\
\\ 
\textbf{Virtualization infrastructure manager} Die zweite Komponente in im Host level management ist der Virtualization infrastructure
manager. Dieser ist für das Zuweisen, Verwalten und Freigeben von virtualisierten (Rechen-, Speicher- und Netzwerkressourcen)
Ressourcen der Virtualisierungsinfrastruktur zuständig. Der Manager ist dabei auch für das Konfigurieren der Infrastruktur für 
ein neues Software Image verantwortlich. Dazu gehört das herunterladen und speicheren der Images. Dabei kann der Manager auch 
Infrastructure-as-a-service Systeme wie Openstack unterstützen. Performance- und Fehlerdaten werden auch der Komponente gesammelt.
\\
Kommt es zu einer Verschiebung einer Anwendung, in der eine laufende Anwendung zu einer externen Cloud umgezogen wird,
iteragiert der Virtualization infrastructure manager mit der externen Cloud um den Handoff an die neue Cloud durchzuführen.
Wie so ein Handoff zwischen zwei Edge Clouds ablaufen kann wird in Kaptiel \ref{x} genauer erläutert.
\subsubsection{MEC system level management:}


\subsection{Beispiel einer Anfrage}

\subsection{Weitere Frameworks}
\label{subsec:Weitere Frameworks}
\section{Herausforderungen durch die Mobilität}

\subsection{Verbindung zwischen dem UE und der Anwendung}
Je nach Anwendung, ist es wichtig, dass die Anwendungen die in der Edge Cloud läuft, 
\subsection{Application State Herausforderungen}
\subsection{Umzug einer Anwendung innerhalb der MEC}
\subsection{Umzug einer Anwendung zu einer externen Cloud}

\subsection{MEC Service Orchestrierung}
\label{subsec:MEC Service Orchestrierung}
\subsection{MEC Mobilität}
\label{subsec:MEC Mobilität}
\subsection{MEC Deployment}
\label{subsec:MEC Deployment}


\section{Anwendungen}
\label{sec:Anwendungen}
\subsection{Internet of Things}
\label{subsec:Internet of Things}
\subsection{Smart Factories}
\label{subsec:Smart Factory}
\subsection{Autonomes Fahren}
\label{subsec:Autonomes Fahren}
\section{Fazit und Ausblick}
Multi Access Edge Cloud Computing ist eine neue Technologie, die die Vorteile von
Edge Clouds und 5G optimial kombiniert und eine neue Platform zur Verfügung stellt.
MEC ist ist eine der wichtigsten neuen Technologien für 5G-Systeme, 
da es zur einer geringen Latenz und zur Kapazitätsverbesserung im Backhaul- und Kernnetzwerk führt. 
Der Erfolg von MEC hängt im Wesentlichen von der Ausrichtung der Technologie auf ETSI NFV ISG ab.
\subsection{Fazit}
\subsection{Ausblick}
Neben dem Ausbau des 5G Netzes, ist bei der Einführung von \textit{MEC} Systemen, 
die Standardisierung sehr wichtig. 
\subsection{Angebote}
\label{subsec:Angebote}
\label{sec:Ausblick}

% -------------------------------------------------------------------------------------------------

% -------------------------------------------------------------------------------------------------
\newpage
% Normaler LNCS Zitierstil
%\bibliographystyle{splncs}
\printbibliography[heading=bibintoc]


\end{document}
